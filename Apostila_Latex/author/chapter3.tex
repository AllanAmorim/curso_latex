%%%%%%%%%%%%%%%%%%%%% chapter.tex %%%%%%%%%%%%%%%%%%%%%%%%%%%%%%%%%
%
% sample chapter
%
% Use this file as a template for your own input.
%
%%%%%%%%%%%%%%%%%%%%%%%% Springer-Verlag %%%%%%%%%%%%%%%%%%%%%%%%%%
%\motto{Use the template \emph{chapter.tex} to style the various elements of your chapter content.}
\chapter{T\'opicos Espec\'ificos}
\label{intro} % Always give a unique label
% use \chaptermark{}
% to alter or adjust the chapter heading in the running head

Tendo em vista o que foi visto no capítulo anterior, é possível perceber que em um texto em \LaTeX{} a formatação é algo muito importante para que o texto final fique da forma desejada. Neste capítulo, será discutido sobre alguns tópicos específicos necessários para a escrita.

\section{Acentua\c c\~ao}
\label{sec:2}
Em \LaTeX, existem duas maneiras de se acentuar as palavras. A primeira forma é manualmente e a segunda é da forma tradicional. Para escrever de forma manual, existe um modelo universal em \LaTeX. Veja o exemplo abaixo:
\begin{trailer}{Acentuação manual}
\begin{verbatim}
\begin{document}
%Colocar  ~, ç
Acentua\c c\~ao  

%Colocar ´, `, ^
S\'ilaba  %aspas simples
\`a       %o acento grave
Jud\^o    %o acento circunflexo

\end{document}\end{verbatim}
\end{trailer}
 \noindent Já para colocar a acentuação de forma direta pelo teclado basta acrescentar no preâmbulo o pacote \verb|\usepackage[utf8]{inputenc}|. 

\section{Se\c c\~oes e subse\c c\~oes}
Dentro do ambiente \verb|document|, podemos usar comandos como \verb|section{}| para criar se\c c\~oes no documento e \verb|subsection{}| para criar subse\c c\~oes. As se\c c\~oes ajudam a organizar o conte\'udo em partes distintas, facilitando a leitura e compreens\~ao. Veja o exemplo:

\begin{trailer}{Seção e subseção}
\begin{verbatim}\begin{document}
    \section{Nome da seção}
    Texto
    
    \subsection{Nome da subseção}
    Texto
\end{document}\end{verbatim}
\end{trailer}
\section{Ambientes de equa\c c\~ao}
Em textos matemáticos é inevitavél a escrita de equações. E uma das vantagens da escrita em \LaTeX{} é o fato das equações ficarem organizadas. No \LaTeX{} podemos utilizar os seguintes ambientes para a escrita matemática: \verb|equation|, \verb|eqnarray|, \verb|array|, \verb|align|, \verb|gather| ou utilizar \verb|$equação$|. Veja um exemplo abaixo:

\begin{trailer}{Ambiente de equação}
No ambiente \verb|equation| não é necessário usar \verb|$$|, pois já é um ambiente matemático.
\begin{verbatim}\begin{document}
% Equação numerada
    \begin{equation} 
    x+y=56
    \end{equation}   
    
% Equação não numerada
    \begin{equation*} 
    x+y=56
    \end{equation*}
\end{document}\end{verbatim}
\end{trailer}

\begin{trailer}{Resultado}
\noindent Equação numerada
    \begin{equation} 
    x+y=56
    \end{equation}   
    
\noindent Equação não numerada
    \begin{equation*} 
    x+y=56
    \end{equation*} 
\end{trailer}

\subsection{Alinhamento de equa\c c\~oes}
\noindent Ao escrever as equações em um texto, às vezes é necessário alinhá-las. No \LaTeX{}, para se alinhar as equações, é preciso utilizar o pacote \verb|\usepackage{amsmath}| e o ambiente \verb|align|.\\
No ambiente \verb|align| não precisa usar \verb|$$|, pois já é um ambiente matemático. Assim, quando temos duas ou mais equações, podemos alinhá-las da seguinte forma: 

\begin{trailer}{Alinhar equação}
\begin{verbatim}\begin{document}
    \begin{align*} 
    2x + 3y &= 7 
    3x - 2y &= 4
    \end{align*}

\end{document}
\end{verbatim}
\end{trailer}

\noindent Obteremos o seguinte resultado: 
\begin{align*}
    2x + 3y &= 7 \\
    3x - 2y &= 4
\end{align*}
% \begin{trailer}{Outro exemplo}
% \noindent 
%     \begin{gather}
%           x+y=56 
%     \end{gather} 
 
% \end{trailer}

\noindent Para especificar onde será o alinhamento na próxima linha se utiliza \verb|&| antes da parte que deseja alinhar

\section{Defini\c c\~ao de novos comandos}
Na escrita matemática podem aparecer Teoremas, Lemas, Corolários, Proposições, entre outros. No \LaTeX{}, é possível definir esses ambientes no preâmbulo e inseri-los ao longo do texto de acordo com a necessidade. Veja o exemplo abaixo:

\begin{trailer}{Novos comandos}

\begin{verbatim}
\newtheorem{definition}{Defini\c c\~ao}[section]
\newtheorem{theorem}{Teorema}[section]
\newtheorem{cor}{Corol\'ario}[section]
\newtheorem{lemma}{Lema}[section]
\newtheorem{prop}{Proposi\c c\~ao}[section]
\newtheorem{Ex}{Exemplo}[section]
\newtheorem{Remark}{Observa\c c\~ao}[section]

\begin{document}
    \begin{theorem}
    Enunciado do teorema 
    \end{theorem}
    
% Para os outros comandos se faz da mesma forma

\end{document}\end{verbatim}
\end{trailer}

\begin{trailer}{Resultado}
   \begin{teorema}
  Enunciado do teorema
   \end{teorema}
\end{trailer}

\eject
\enlargethispage{24pt}



